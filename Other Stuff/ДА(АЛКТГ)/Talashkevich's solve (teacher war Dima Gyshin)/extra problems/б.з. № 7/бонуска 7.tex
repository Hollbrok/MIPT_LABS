
\documentclass[a4paper,12pt]{article} % тип документа


% Русский язык
\usepackage[T2A]{fontenc} % кодировка
\usepackage[utf8]{inputenc} % кодировка исходного текста
\usepackage[english,russian]{babel} % локализация и переносы


% Математика
\usepackage{amsmath,amsfonts,amssymb,amsthm,mathtools}


\usepackage{wasysym}

%Заговолок
\author{Талашкевич Даниил Александрович}

\title{Бонусная задача. Неделя № 7}

\date{\today}

\begin{document}

\maketitle
\thispagestyle{empty}

\newpage
\setcounter{page}{1}
\begin{center}
\itshape
\bfseries
{ \Large БОНУСНАЯ ЗАДАЧА № 	7}
\end{center}

{\bf (*) } Найдите число треугольников с целочисленными сторонами и периметром $N$. (Подсказка: ответ зависит от остатка $N$ при делениина $12$).
\begin{center}
\bfseries
{\Large Решение: }
\end{center}

Без ограничений общности $a\leqslant b \leqslant c$ и должно выполняться условие $a+b > c$.

Ясно, что $c$ может принимать значения от $\left[ \frac{N}{3}\right] + 1= c_0$ до $\left[ \frac{N+1}{2}\right]  - 1 = c_1$. Так же обозначим $ $ .Для каждого значения $c$ выполняется $a+b = 100 - c$, значит $b$ может принимать значения от $\frac{N - c}{2}$,а для четных-нечетных $c$ :

1) от $\frac{N}{2} - \frac{c}{2}$ -- при четном $c$. Всего $\frac{3c}{2} - \frac{N}{2} - 1$ вариантов.

2) от $50 - \frac{c-1}{2}$ до $c$ -- при нечетном $c$. Всего $\frac{3c-1}{2} - \frac{N}{2} - 1$.

При $c$ равным $\left[ \frac{N}{3}\right]  + 1 +2n(n \in \mathbb{N})$ получаем :
\[ \sum\limits_{c = c_0}^{c_1} \frac{3c}{2} - \left[ \frac{N+1}{2}\right]  - 1 \]


\begin{flushright}
\begin{large}
\textbf {Доказано }
\end{large}
\end{flushright}


\end{document}


