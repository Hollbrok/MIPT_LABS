
\documentclass[a4paper,12pt]{article} % тип документа


% Русский язык
\usepackage[T2A]{fontenc} % кодировка
\usepackage[utf8]{inputenc} % кодировка исходного текста
\usepackage[english,russian]{babel} % локализация и переносы


% Математика
\usepackage{amsmath,amsfonts,amssymb,amsthm,mathtools}


\usepackage{wasysym}

%Заговолок
\author{Талашкевич Даниил Александрович}

\title{Неделя 11. Ориентированные графы
и отношения порядка}

\date{\today}

\begin{document}

\maketitle
\thispagestyle{empty}

\newpage
\setcounter{page}{1}
\begin{center}
\itshape
\bfseries
{ \Large Problems:}
\end{center}

{\bf 1.} Известно, что в неориентированном графе существует маршрут,
проходящий по каждому ребру ровно два раза. Верно ли, что в графе
есть замкнутый эйлеров маршрут?
\begin{center}
\bfseries
{\Large Решение: }
\end{center}

Замкнутого Эйлерова маршрута не будет, если будет вершина с нечетной степенью. Приведем пример такого случая :	возьмем полный граф $K_4 \setminus {AC}$, то есть уберем диагональ. Тогда у нас существует маршрут, проходящий по каждому ребру ровно два раза $ABDCBADCBDA$, Но Эйлерова маршрута не будет в этом случае, так как есть вершина со степенью $3$.

\begin{flushright}
\begin{large}
\textbf {Ответ: Неверно.}
\end{large}
\end{flushright}

{\bf 2.} Выходная (она же исходящая) степень каждой вершины в ориен тированном графе на $n > 3$ вершинах равна $n - 2$. Какое количество компонент сильной связности может быть в этом графе? Укажите все возможные значения.
\begin{center}
\bfseries
{\Large Решение: }
\end{center}

Логические рассуждения следующие: предположим, что в графе больше, чем одна компонента сильной связности. Тогда существуют две такие вершины графа $X$ и $Y$, что нет пути из $X$ в $Y$. Так как всего $n$ вершин и исходящая степень каждой вершины равна $n-2$, то вершина $X$ соединена с любой другой вершиной (есть путь к любой вершине), кроме $Y$. 

Пусть вершина $X$ входит в первую компоненту сильной связности, а $Y$ во вторую компоненту связности. Из вершины $X$ можно добраться во все вершины, кроме $Y$, а значит если из любой из остальных вершин есть путь к $Y$, то есть путь из $X$ в $Y$, что неверно по предположению -- нет пути ни из одной вершины в вершину $Y$. Значит вторая компонента связности состоит из единственной вершины, от которой есть путь к $n-2$ вершинам из первой компоненты сильной связности.

Значит если компонент сильной связности и больше, чем $1$, то значит их может быть только $2$, а пример для единственной компоненты сильной связности -- неориентированный граф цикл длины $4$ (между двумя верщинами есть двустороння достижимость).

\begin{flushright}
\begin{large}
\textbf {Ответ: $1$ или $2$ компоненты}
\end{large}
\end{flushright}

{\bf 3.} Пусть в ориентированном графе для любой пары вершин $u, v$ есть
либо ребро $(u, v)$, либо ребро $(v, u)$ (ровно одно из двух). Докажите, что
в таком графе есть (простой) путь, включающий в себя все вершины.
\begin{center}
\bfseries
{\Large Решение: }
\end{center}

Доказательство приведем по индукции:

$1)$ Для кол-ва вершин $|V| = 2$ очевидно, что выполняется. (для $|V| = 1$ тоже).

$2)$ Пусть при $|V| = n$ существует простой путь длины $n$, включающий в себя все веришины:

\[ A\rightarrow B \rightarrow C \rightarrow ... \rightarrow N.\]

$3)$ Докажем, что при $|V| = n + 1$  выполняется (новую вершину обозначим за $M$):

Если у нас существует один из путей $M \rightarrow A$ или $N \rightarrow M$, тогда, очевидно, что у нас будет существовать простой путь длины $n+1$ такой, что включает в себя все вершины.

Рассмотрим случай, когда ни один из этих путей не существует, тогда , по условию, у нас одновременно существуют два других путя $A\rightarrow M$ и $M \rightarrow N$ (так как по условию любые пары верших имеет одно из двух возможных ребер).

Теперь, если у нас существует путь $M \rightarrow B$, то мы можем выбрать путь $A\rightarrow M \rightarrow B \rightarrow C \rightarrow ... \rightarrow N$ и будет выполняться неоходимое условие. Если же такого путя не существует, то можем проделать аналогичную операцию с последней из верших (обозначим ее за $X$, а следующую за $Y$) 

\[ A\rightarrow B \rightarrow C \rightarrow ...\rightarrow X \rightarrow M \rightarrow Y \rightarrow...\rightarrow N.  \]

Если же $\forall\ u \in |V|\ \nexists X : \exists (X,M)$, то тогда путь можно выбрать как $ A\rightarrow M \rightarrow B \rightarrow C \rightarrow ... \rightarrow N.$ 

Доказано по индукции.


\begin{flushright}
\begin{large}
\textbf {Ответ: доказано.}
\end{large}
\end{flushright}

{\bf 4.} Профессор Рассеянный построил частичный порядок <P для утрен-
него одевания:

\begin{flushleft}

$ \textbf{очки} <_p \textbf{брюки} <_p \textbf{ремень} <_p \textbf{пиджак};$

$ \textbf{очки} <_p \textbf{рубашка} <_p \textbf{галстук} <_p \textbf{пиджак}; $

$ \textbf{брюки} <_p \textbf{туфли}; $

$ \textbf{очки} <_p \textbf{носки} <_p \textbf{туфли};  $

$ \textbf{очки} <_p \textbf{часы}; $

\end{flushleft}

\begin{center}
\bfseries
{\Large Решение: }
\end{center}

Понятно, что первый элемент -- очки, так как они надеваются раньше всех вещей (очевидно из условия). Из первых двух условий построим такой порядок:
$\text{очки} <_P \text{брюки} <_P \text{ремень} <_P \text{рубашка} <_P \text{галстук}<_P \text{пиджак}$. Туфли стоят после брюк и носков, значит носки ставим в любое место после брюк, а за ними в любом месте туфли: $\text{очки} <_P \text{брюки}<_P \text{носки} <_P \text{туфли}<_P \text{ремень} <_P \text{рубашка} <_P \text{галстук}<_P \text{пиджак}$. Часы можно добавить в любое место после очков: $\text{очки} <_P \text{брюки} <_P \text{носки} <_P \text{туфли} <_P \text{ремень} <_P \text{рубашка} <_P$
$ <_P \text{галстук} <_P \text{пиджак} <_P \text{часы}$

\begin{flushright}
\begin{large}
\textbf {Ответ: $\text{очки} <_P \text{брюки} <_P \text{носки} <_P \text{туфли} <_P \text{ремень} <_P \text{рубашка} <_P <_P \text{галстук} <_P \text{пиджак} <_P \text{часы}$}
\end{large}
\end{flushright}

{\bf 5.} В Вестеросе $n$ городов, каждые два соединены дорогой. Дороги

сходятся лишь в городах (нет перекрестков, одна дорога поднята эста-
кадой над другой). Злой волшебник хочет установить на всех дорогах

одностороннее движение так, что если из города можно выехать, то в
него нельзя вернуться. Докажите, что

{\bf а)} Волшебник может это сделать;

{\bf б)} Найдется город, из которого можно добраться до всех, и найдется
город, из которого нельзя выехать;

{\bf в)} Существует единственный маршрут, обходящий все города.

{\bf г)} Сколькими способами волшебник может осуществить свое намере-
ние?
\begin{center}
\bfseries
{\Large Решение: }
\end{center}

{\bf а)} Так как мы можем пронумеровать города(вершины графа) числами от $1$ до $|V|$ так, что ребра графа $G$ идут только от вершины с меньшим номером в вершины с большим номером. А так это эквивалентно тому, что граф $G$ ациклический, то значит, что будет выполняться условие односторонности движения.

{\bf б)} Так как в ациклическом ориентированном графе есть сток( в нашей задаче -- это город, из которого нельзя выехать), то получаем, что такое город есть. Доказательство:

Возьмём самый длинный ориентированный путь $v_1 \rightarrow v_2 \rightarrow . . . \rightarrow v_n$ в графе $G$. Такой существует, потому что множество путей конечно
(вершины в пути повторяться не могут). Докажем от противного, что вершина $v_n$
является стоком $(d_{+}(v_n) = 0)$. Если в $G$ есть ребро $v_n \rightarrow u$ и $u \neq v_i$ , то путь можно сделать длиннее, добавив к нему $u$; если же $u = v_i$, то в графе $G$ есть цикл (петель в $G$ по определению быть не может). Оба случая приводят нас к противоречию.

Доказательство того, что есть существует город, из которого можно добраться до всех остальных можно доказать по индукции (так же доказывалось на лекции).

Заметим, что сток имеет наибольший номер, а исток наименьший.

{\bf в)} Так как построение этого графа было по принципу "из меньшего номера в больший", то существует только один маршрут, обходящий все города: $1\rightarrow 2 \rightarrow ... \rightarrow n$.

{\bf г)} Так как осуществление намерений волшебника зависит только от того, как мы пронумеруем города: первый город -- $n$ способов, второй -- $n - 1$ и т.д. Получаем способов реализовать одностороннее движение $n!$.

\begin{flushright}
\begin{large}
\textbf {Ответ: ответы в решении.}
\end{large}
\end{flushright}

{\bf 6.} Бинарное отношение $P$ называется турниром, если оно антирефлексивно, антисимметрично и линейно. (Неформально — это результат кругового турнира — каждую альтернативу сравнили с каждой и запомнили результат). Докажите, что либо турнир -- строгий линейный порядок, либо существуют такие альтернативы $a, b, c$ , что $aP b, bP c$ и $cP a.$
\begin{center}
\bfseries
{\Large Решение: }
\end{center}

Предположим, что бинарное отношение не является строгим линейным порядком, тогда, так как оно линейное, антирефлексивное и антисимметричное, оно не является транзитивным, то есть есть набор элементов $a, b, c$ таких, что $aPb \wedge bPc \nRightarrow aRc$. Так как по условию бинарное отношение линейно, то существует хотя бы одна из пар $(a,c), (c,a)$ в бинарном отношении, значит существует пара $(c,a)$ и выполняется $aPb, bPc, cRa$.

Теперь пусть не выполняется второе условие задачи на существование альтернатив, то есть нет таких $a, b, c$, что $aPb, bPc, cRa$. Выберем три случайных элемента $a, b, c$ так, чтобы $aRb$. Рассматрим 2 возможные ситуации: выполняется $bRc$ или выполняется $сRb$. Из первой ситуации следует, что выполняется $cRa$, то есть для бинарного отношения, определённого на множестве из $3$ элементов, для которые выполняются свойства в условии задачи, в первом случае всегда выполняется транзитивность. Вторая ситуация: выполняется $aRc$, аналогично для бинарного отношения из условий, определённого на множестве из $3$ элементов всегда выполняется транзитивность. Так как для любых бинарных отношений на $3$ элементах выполняется транзитивность, то и для всегда рассматриваемого бинарного отношения выполняется транзитивность (можно доказать от противного). Значит бинарное отношение является строгим линейным порядком.

\begin{flushright}
\begin{large}
\textbf {Ответ: 15}
\end{large}
\end{flushright}

{\bf 7.} Сколько есть порядков на $n$-элементном множестве, в которых ровно
одна пара элементов несравнима?
\begin{center}
\bfseries
{\Large Решение: }
\end{center}

Пусть $A_i$ и $A_j$ несравнимы, тогда по условию это единственная пара и все элементы без $A_i$ или $A_j$ образуют линейный порядок. Вполне очевидно, что $A_i$ и $A_j$ расположены рядом( то есть между одними и теми же элементами). Если же это не так, то получаем противоречие: 

\[ A_{i - 1} < A_i < A_{i + 1}\ \wedge \ A_{i - 3} < A_j <A_{i - 1} \Rightarrow  A{j} < A_{i-1} < A_i - \textbf{противоречие.}\]

Тогда всего порядков: 

1) Выбираем несравнимую пару $C_n^2$

2) Задаем линейный порядок из остальный $(n - 2)$ элементов : $(n - 2)!$

3) Выбираем где можно расположить выбранную пару: ($n - 1$) способов.

Получаем в итоге кол-во способов: $\frac{(n - 1) n!}{2}.$ 

\begin{flushright}
\begin{large}
\textbf {Ответ: $\frac{(n - 1) n!}{2}.$}
\end{large}
\end{flushright}

{\bf 8.} Докажите, что любой частичный порядок $P$ на конечном множестве $A$ можно продолжить до линейного. То есть можно добавить в $P$ некоторые пары элементов из $A \times A$ так, что любые два элемента $a, b \in A$ окажутся сравнимы: будет выполнено либо $aPb$ либо $bPa$.
\begin{center}
\bfseries
{\Large Решение: }
\end{center}

Во-первых, если рассматриваемое отношение частичного порядка антирефлексивно, то не будем в него добавлять ни одну пару из $A \times A$ вида $(a,a)$, а если рефлексивно, то сразу дополним его всеми такими парами.

Дополнять отношение частичного порядка надо теперь так, чтобы сохранялись антисимметричность и транзитивность, то есть условно надо связать единственным образом $2$ элемента, которые не были связаны до этого, так, чтобы сохранилась транзитивность.

Понятно, что в зависимости от рефлексивности и антирефлексивности отношение частичного порядка будет срогим или нестрогим, и что каждому отношению нестрогого порядка соответствует единственное отношение строго порядка, которое можно получить удалением всех пар вида $(a,a)$. По лемме отношению строгого порядка соответствует ориентированный ациклический граф.

Тогда отношение строго порядка явно можно дополнить всеми недостающими парами так, чтобы когда мы пронумеровали все элементы ориентированного ациклического графа, то будет выполняться $a_1 < a_2 < ... < a_{|A|}$ (символ $<$ соответсвует отношению строгого порядка), так как всегда можно связать два элемента символами $<$ или $>$, и если получился символ $>$, то меняем номера элементов местами, а транзитивность сохранится.

Если отношение строго порядка было рефлексивным, то, как уже было сказано, добавляем в него все возможные пары $(a,a)$, чтобы и новое линейное отношение частичного порядка стало рефлексиным, если отношение частичного порядка было антирефлексивным, то ничего не добавляем.


\begin{flushright}
\begin{large}
\textbf {Ответ: доказано}
\end{large}
\end{flushright}

{\bf 9.} Граф $G$ имеет множество вершин $V  =\{ 1,2,3,5,6,10,15,30 \} $. Граф $G$ содержит ребро $\{u, v \}$ (для определённости $u < v$), если $v$ делится на $u$ и не существует (отличной от $u$ и $v$) вершины $s \in V$, такой что и $v$ делится на $s$ и $s$ делится на $u$.

{\bf 1)} Постройте граф $G$.

{\bf 2)} Изоморфен ли этот граф булеву кубу $B_3$? При положительном ответе укажите биекцию.
\begin{center}
\bfseries
{\Large Решение: }
\end{center}

{\bf 1)}Так как граф $G$ содержит только определенные ребра, то получаем граф $G :$
\[E = \{ (1,2), (1,3), (1,5),(2,6), (2,10),(3,6),(3,15),\]
\[(5,10), (5,15), (6,30), (10,30), (15,30)\} .\]

{\bf 2)} Из "рисунка"\  видно, что можно построить следующую биекцию:

Для вершины булева куба $\overline{abc}$ поставим в соответствие вершину нашего графа $G$ так, что:

$\overline{abc} \rightarrow 2^a\cdot 3^b\cdot 5^c$ (к примеру $010 \rightarrow 3$).

Построена биекция.



\begin{flushright}
\begin{large}
\textbf {Ответ: ответы в решении.}
\end{large}
\end{flushright}

{\bf 10.} 
\begin{center}
\bfseries
{\Large Решение: }
\end{center}

\begin{flushright}
\begin{large}
\textbf {Ответ: }
\end{large}
\end{flushright}

\end{document}


