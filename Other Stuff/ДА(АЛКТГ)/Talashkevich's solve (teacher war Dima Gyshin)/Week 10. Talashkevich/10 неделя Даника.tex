	
\documentclass[a4paper,12pt]{article} % тип документа


% Русский язык
\usepackage[T2A]{fontenc} % кодировка
\usepackage[utf8]{inputenc} % кодировка исходного текста
\usepackage[english,russian]{babel} % локализация и переносы


% Математика
\usepackage{amsmath,amsfonts,amssymb,amsthm,mathtools}


\usepackage{wasysym}

%Заговолок
\author{Талашкевич Даниил Александрович}

\title{Неделя № 10. Бинарные отношения и их графы.
Отношения эквивалентности}

\date{\today}

\begin{document}

\maketitle
\thispagestyle{empty}

\newpage
\setcounter{page}{1}
\begin{center}
\itshape
\bfseries
{ \Large Problems:}
\end{center}

{\bf 1.} Ответьте на следующие вопросы для бинарного отношения $R \subseteq \{1, 2, 3\}\times\{1, 2, 3\}$. Является ли $R$ рефлексивным? симметричным? транзитивным? отношением эквивалентности? Для каждого отношения $R$ нарисуйте соответствующий граф. Используйте неориентированный граф для симметричных бинарных отношений, в случае нерефлексивных бинарных отношений используйте петли.

{\bf а)} $R = \{ (1,1), (2,2), (3,3), (1,2), (1,3), (3,2)\}$.

{\bf б)} $R = \{ (1,1), (1,2), (2,1), (2,2)\}$
\begin{center}
\bfseries
{\Large Решение: }
\end{center}

{\bf а)} $R = \{ (1,1),(2,2),(3,3),(1,2),(1,3),(3,2)\};$

Так как $R \subseteq (1,2,) \times (1,2,3)$, то $R$ является рефлексивным благодаря первым трем парам. Так же не является симметричным хотя бы потому, что нет пары $(2,1)$ для пары $(1,2)$. И так же является транзитивным. Значит отношением эквивалентности не является.

{\bf б)} $R = \{(1,1), (1,2), (2,1), (2,2) \}$

Не рефлексивно, потому что нет пары $(3,3)$. Симметрично и транзитивно , значит это отношение эквивалентности.

\begin{flushright}
\begin{large}
\textbf {Ответ: а) отношение эквивалентности. б) нет.}
\end{large}
\end{flushright}

{\bf 2.} Выразите отношение «племянник(-ца)» через отношения «отец» и
«мать» и операции над отношениями. 
\begin{center}
\bfseries
{\Large Решение: }
\end{center}

$\exists C,D:$ $A$ племянник $B$ $\Leftrightarrow$ (($B$ брат $C) \vee (B$ сестра $C$))$\wedge ((C$ отец $A) \vee (C$ мать $A$)) $\Leftrightarrow  \exists C,D:$ $(((D$ отец $B) \wedge (D$ отец $C)) \vee ((D$ мать $B) \wedge (D$ мать $C))) \wedge ((C$ отец $A) \vee (C$ мать $A))$. 

\begin{flushright}
\begin{large}
\textbf {Ответ: $\exists C,D:$ $(((D$ отец $B) \wedge (D$ отец $C)) \vee ((D$ мать $B) \wedge (D$ мать $C))) \wedge ((C$ отец $A) \vee (C$ мать $A))$. }
\end{large}
\end{flushright}

{\bf 3.} Пусть бинарные отношения $P_1,P_2 \subseteq A \times A$ транзитивны. Будут ли $\overline{P_1}, P_1 \cap P_2, P_1 \cup P_2, P_1 \circ P_2$ обладать теми же свойствами. 
\begin{center}
\bfseries
{\Large Решение: }
\end{center}

1) $\overline{P_1}$. Приведем контрпример, когда $P_1$ транзитивно, а $\overline{P_1}$ нет:

Транзитивность можно изобразить как граф-треугольник $K_3$ и изолированная вершина. Транзитивность будет выполнятся если есть путь длины 1 между вершинами $a,b$, а с $ b $ до другой вершины $c$ существует путь длины 1, то тогда должен существовать путь длины 1 между вершинами $a,c$. В свою очередь дополнение этого графа будет не транзитивно, потому что найдутся в этом дополнении две вершины, между которыми на будет прямого пути.

Отсюда ответ: $\overline{P_1}$ не обладает транзитивностью

2) $P_1 \cap P_2$. Рассмотрим $x,y,z$: если $(x;y)$ и $(y;z)$ входит и в $P_1$ и в $P_2$(мы рассматриваем числа, которые принадлежат $P_1 \cap P_2$), то по транзитивности следует, что $(x;z)$ входит. Но тогда будет справедливо, что и $(x;z)$ входит и в $P_1$ и в $P_2$ (по транзитивность), тогда и $(x;z)$ входит в $P_1 \cap P_2$. 

$P_1 \cap P_2$ -- обладает транзитивностью, если $P_1$ и $P_2$ обладают.

3) $P_1 \cup P_2$. контрпример: возьмем $P_1,P_2$ такие же, как и в пункте 1, которые пересекаются в одной вершине, тогда получаем: ($1;2) \wedge (2;3) \rightarrow (1;3)$. Но для $P_1 \cup P_2$ транзитивность не выполняется, потому что нет пути длины 1 между вершинами (1,3).

4) $P_1 \circ P_2$. Приведем контрпример. $P_1 = \{(1,2),(3,4)\}$ -- транзитивность выполняется, $P_2 = \{(2,3),(4,5)\}$ -- транзитивность так же выполнятся. Рассчитаем тогда их композицию $P_1 \circ P_2 = \{(1,3),(3,5) \}$ -- транзитивность не выполняется, т.к. нет пары ($(1,5)$). Приведен контрпример , значит транзитивность не выполняется.

\begin{flushright}
\begin{large}
\textbf {Ответ: ответы в решении.}
\end{large}
\end{flushright}

{\bf 4.} Бинарное отношение на множестве из $6$ элементов содержит $33$ пары.
Может ли оно быть {\bf а)} симметричным; {\bf б)} транзитивным?
\begin{center}
\bfseries
{\Large Решение: }
\end{center}

а) В частном случае да, могут. Всего пар $6^2 = 36$, из них вида $(a,a)$, где $a \in A$ всего 6 по условию, тогда остается 30 пар. Если мы всех их используем, то получается, то будет выполнятся симметричность для 30 пар. А в качестве остальных 3 можем взять любых 3 пары вида $(a,a)$.

б) Аналогичный пример как и в пункте {\bf а)}. Берем все пары, кроме 3 вида $(a,a)$.

\begin{flushright}
\begin{large}
\textbf {Ответ: а,б) да.}
\end{large}
\end{flushright}

{\bf 5.} Какие из следующих бинарных отношений на множестве $\mathbb{N}$ — отношения эквивалентности?

{\bf a)} $xPy$: у чисел $x$ и $y$ одинакова последняя цифра (здесь и далее в десятичной записи).

{\bf б)} $xQy$ : числа $x$ и $y$ отличаются в ровно одной цифре.

{\bf в)} $xRy$ : разница между суммой цифр $S_x$ и $S_y$ четна. Формально: пусть $\overline{x_nx_{n-1}...x_1x_0}$ -- десятичная запись числа $x$; тогда $S_x = \sum\limits_{k = 0}^{n}x_k.$  
\begin{center}
\bfseries
{\Large Решение: }
\end{center}

{\bf a)} рефлексивность: так как у одинаковых чисел одинаковые числа на конце $\Rightarrow$ выполняяется рефлексивность. Симметричность : если $a$ и $b$ имею одинаковые числа на конце, то очевидно, что выполняется симметричность. Транзитивность: если $a$,$b$ имеют одинаковое числа на конце,пусть это число $x$, тогда, если выполняется, что у числа $c$ на конце $x$, то,очевидно, что  $\Rightarrow$ выполняется транзитивность.

Получаем, что это отношение эквивалентно.

{\bf б)} рефлективность: у двух одинаковых чисел не могут быть разные числа на конце $\Rightarrow$ рефлективность не выполняется $\Rightarrow$ это отношение не эквивалетно.

{\bf в)} Рефлективность, очевидно, выполняется, так как ($S_x - S_y = 0$) -- четное число. Симметричность: Если ($S_x - S_y $) -- четное, то, очевидно, что и ($S_y - S_x$) -- четное, так как оба числа $x$ и $-x$ могут быть четными. Транзитивность: если ($S_y - S_x$) -- четное и пусть равняется $2k$, $k \in \mathbb{N} $ и $S_z - S_x $ -- четное и пусть равно $2p$ $\Rightarrow$ $2p-2k = 2(p-k)$ -- что, очевидно, четное число $\Rightarrow$ выполняется транзитивность, а значит это отношение эквивалентно. 

\begin{flushright}
\begin{large}
\textbf {Ответ: {\bf a),б)} -- эти отношения эквиваленты; {\bf в)} нет.}
\end{large}
\end{flushright}

{\bf 6.}  Найдите число отношений эквивалентности на множестве $\{ 1,2,3,4\}.$
\begin{center}
\bfseries
{\Large Решение: }
\end{center}

Общее количество отношений эквивалентности на $\{ 1 , 2 , 3 , 4 \} $= количество разделов набора $\{ 1 , 2 , 3 , 4 \}$ на классы эквивалентности (непустые подмножества) такие, что их пересечение пусто и их объединение дает $\{1,2,3,4\}$. 

Количество возможных перегородок в комплекте $A = $ Номер Белла, $B_n = \sum_{k = 0}^n \left \{{n \atop k} \right \}$ -- Сумма чисел Стирлинга второго рода.
Тогда получаем ответ для $n = 4$ число отношений эквивалентности на множестве $\{1,2,3,4\}$  $= 15$.



\begin{flushright}
\begin{large}
\textbf {Ответ: 15}
\end{large}
\end{flushright}

{\bf 7.} Об отображениях (всюду определенных функциях) $f, g$ из множества
$A$ в себя известно, что $f \circ g \circ f = id_A$. Верно ли, что $f$ -- биекция?
(Множество $A$ не обязательно конечное.)
\begin{center}
\bfseries
{\Large Решение: }
\end{center}

От противного: пусть $f$ -- не биекция, тогда так как функция всюду определенная, то она сюръекция, а значит существеют такие $x_1,x_2$, что выполняется $f(x_1) = f(x_2)$. Тогда из первого условия композиции имеем: 
\[ x_1 = id_A(x_1) = (f\circ g\circ f)(x_1) = f(g(f(x_1))) = f(g(f(x_2))) =\]
\[= (f\circ g\circ f)(x_2) = id_A(x_2) = x_2.\]

Значит функция инъективна. Получено противоречие $\Rightarrow f$ биекция.	 

\begin{flushright}
\begin{large}
\textbf {Ответ: верно.}
\end{large}
\end{flushright}

{\bf 8.} Пусть $R$ -- отношение эквивалентности на множестве $A$. Докажите,
что существуют такие множество $B$ и отображение $f : A \rightarrow B$, что
каждый класс эквивалентности $C$ представим в виде $C = f^{-1}(b)$ для некоторого элемента $b \in B.$
\begin{center}
\bfseries
{\Large Решение: }
\end{center}

Разобьем $A$ на части, где каждая часть -- это класс эквивалентности. Возьмем множество $B$ такое, что каждому классу соответствует ровно один элемент $b$. Тогда для каждого класса эквивалентности верно, что $C = f^{-1}(b)$, для некоторого элемента $b \in B.$

\begin{flushright}
\begin{large}
\textbf {Ответ: доказано.}
\end{large}
\end{flushright}

{\bf 9.} Множество $A$ состоит из семи элементов. Найдите количество отоб-
ражений $f : A \rightarrow A$, таких что $f \circ f = id_A$.
\begin{center}
\bfseries
{\Large Решение: }
\end{center}

Обозначим элемент $x$ подвижным, если $f(x)=x$, при этом всегда $f(f(x))=x$ действует на $x$ тождественно по условию. Пусть $f(x)=y$, где $y$ не равно $x$. Тогда $f(y)=x$, то есть остальные элементы разбиваются на пары переходящих друг в друга элементов, а пары всегда являются чётным количеством элементов.

Не неподвижных элементов всегда чётное число, значит неподвижных -- нечётное. Могут быть $1$ неподвижное число, $3$, $5$ и $7$, так как всего есть $7$ элементов.
В последнем случае отображение одно (все $7$ элементов неподвижны) Во третьем случае выбираем $2$ подвижных элемента из семи, которые будут переходить друг в друга, при этом из 2 подвижных элементов можно составить только одну пару. Это $C_7^2=21$ способ. При $3$ неподвижных, выбираем их $C_7^3=35$ способами, при этом остаётся $4$ подвижных элемента, которые $3$ способами разбиваются на пары. Итого $105$ вариантов. Наконец, для первого случая есть $7$ способов выбрать неподвижный элемент. Остаётся $6$ элементов, которые разбиваются на пары $15$ способами (для первого элемента есть $5$ способов выбрать какой-то элемент, с учётом этого для второго элемента есть $4$ способа, для третьего $3$ способа и т. д., значит $1+2+3+4+5 = 15$). Итого $105$ способов, как и в прошлом случае.

В ответе будет $1+21+105+105=232$ отображения.

\begin{flushright}
\begin{large}
\textbf {Ответ: $232.$}
\end{large}
\end{flushright}

{\bf 10.} Пусть $f$ отображение из $\mathbb{Z}^2$ в $\mathbb{Z}$ такое, что \[f(a,b) = a - b.\]

Инъективно ли $f$? Сюръективно ли $f$? Верно ли, что прообраз числа $5$ содержит три элемента из $\mathbb{Z}^2$?
\begin{center}
\bfseries
{\Large Решение: }
\end{center}

$1)$Инъективно? Если $f$ инъективно, то  каждому $c = f(a,b)$ соответствует не более, чем один набор из $a,b$. Приведем контрпример:

$a = b = 1 \Rightarrow f(a,b) = 0$ и $a = b = 2 \Rightarrow f(a,b) = 0$. Значит получаем, что $f$ -- не инъекция.

$2)$Сюръективно? Если $f$ сюъективно, то $\forall c \ \exists a,b : f(a,b) = c$. Рассмотрим данную ситуацию:

Чтобы получить $c$ должно выполняться, что $a - b = c$. Так как $A \in \mathbb{Z}$, то возьмем $a = 2$, тогда по принципу Архимеда найдется $b$ такое, что $b = 2 - c$. Значит $f$ -- сюръективно.

$3)$ Рассмотрим как можно получить число $5$ в этом случае:

$5 = 1+4=2+3=0+5=6+(-1) ..$. Получаем, что прообраз числа $5$ содержит элементы :$\ 0,1,2,3,4,5\ ...$.

Значит прообраз числа $5$ не содержит ровно 3 элемента из $\mathbb{Z}^2$.


\begin{flushright}
\begin{large}
\textbf {Ответ: ответы в решении.}
\end{large}
\end{flushright}

\end{document}


