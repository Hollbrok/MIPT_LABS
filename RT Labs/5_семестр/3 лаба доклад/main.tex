\documentclass[a4paper, 12pt]{article}%тип документа

%%%Библиотеки
	%\usepackage[warn]{mathtext}	
	\usepackage[T2A]{fontenc} % кодировка
	\usepackage[utf8]{inputenc} % кодировка исходного текста
	\usepackage[english,russian]{babel} % локализация и переносы
	\usepackage{caption}
	\usepackage{listings}
	\usepackage{amsmath,amsfonts,amssymb,amsthm,mathtools}
	\usepackage{wasysym}
	\usepackage{graphicx}%Вставка картинок правильная
	\usepackage{float}%"Плавающие" картинки
	\usepackage{wrapfig}%Обтекание фигур (таблиц, картинок и прочего)
	\usepackage{fancyhdr} %загрузим пакет
	\usepackage{lscape}
	\usepackage{xcolor}
	\usepackage[normalem]{ulem}
	\usepackage{hyperref}

%%%Конец библиотек




%%%Настройка ссылок
	\hypersetup
	{
		colorlinks=true,
		linkcolor=blue,
		filecolor=magenta,
		urlcolor=blue
	}
%%%Конец настройки ссылок


%%%Настройка колонтитулы
	\pagestyle{fancy}
	\fancyhead{}
	\fancyhead[L]{Доклад}
	\fancyhead[R]{Талашкевич Даниил, группа Б01-008}
	\fancyfoot[C]{\thepage}
%%%конец настройки колонтитулы



							\begin{document}
						%%%%Начало документа%%%%


%%%Начало титульника
\begin{titlepage}

	\newpage
	\begin{center}
		\normalsize Московский физико-технический институт \\(госудраственный 			университет)
	\end{center}

	\vspace{6em}

	\begin{center}
		\Large Доклад по аналоговой электронике\\
	\end{center}

	\vspace{1em}

	\begin{center}
		\large \textbf{Речевой этикет}
	\end{center}

	\vspace{2em}

	\begin{center}
		\large Талашкевич Даниил Александрович\\
		Группа Б01-008
	\end{center}

	\vspace{\fill}

	\begin{center}
	Долгопрудный \\2022
	\end{center}
	
\end{titlepage}
%%%Конец Титульника



%%%Настройка оглавления и нумерации страниц
	\thispagestyle{empty}
	\newpage
	\tableofcontents
	\newpage
	\setcounter{page}{1}
%%%Настройка оглавления и нумерации страниц


					%%%%%%Начало работы с текстом%%%%%%

\section{Вступление}

Хорошие манеры – это отличительная черта умных людей. Но какие манеры хорошие, а какие плохие? Речевой этикет рассказывает о хороших манерах в речи, которые помогут уверенно общаться с людьми.

Речевой этикет – это советы по уважительному общению с другими людьми. Именно он рассказывает, как правильно общаться со старшими, коллегами, как отвечать на неловкие вопросы. Все правила сводятся к формулам речевого этикета.

Правила общения касаются встречи (знакомства), общения при разговоре и его завершении. Они применимы к устной и письменной речи, официальным и речевым обращениям.

\section{Функции речевого этикета}
Речевой этикет делает общение приятным. Он нужен для вежливой беседы, правильным обращениям к старшим и руководящим должностям. Функции речевого этикета зависят от формы общения:

\begin{itemize}
\item Слова и выражения для приветствий и прощаний, слова вежливости;
\item Обогащение речи, исключение слов-паразитов, мычаний и пауз;
\item Решение конфликтов спокойной и вежливой беседой;
\item Умение слушать собеседника до конца, не прерывая его вопросами и своими доводами. 
\end{itemize} 

\section{История}
Речевой этикет появился давно, когда люди только собирались в племена. Уже тогда к главам поселений и лекарям применялись вежливые формы обращения. К вождям, лекарям, воинам, жрецам были свои обращения, которые сохранились и сегодня.

Первым речевым этикетом были приветствия. Племена танцевали перед другими племенами, наклонялись или делали другие жесты. В Китае и Японии наклонялись со сжатыми ладонями, на Руси делали наклоны, и чем глубже, тем больше уважения было в жесте. Сейчас люди по всему миру жмут руки, целуют друг другу щеки, обнимают и похлопывают по спине.

Особенно популярны правила речевого поведения были у знати в XVII-XIX веках. После Октябрьской Революции универсальным вежливым обращением стали «товарищ» и «гражданин». До революции употребляли слова барин, барышня, государь. За границей популярны были слова сэр, милорд. Сейчас в уважительной форме принято говорить мисс, миссис, мистер, доктор и пр.

Сейчас в России и странах СНГ нет особых обращений. К незнакомым людям принято обращаться на «вы», «молодой человек», «девушка», «женщина», «мужчина».

\section{Правила}

Соблюдать правила речевого этикета просто и нужно, красивая и правильная речь вызывает симпатию у собеседника.

Вот самые простые правила речевого этикета:

\begin{itemize}

\item Здороваться в полной форме: не «здрасьте», а «здравствуйте», употреблять слова добрый день и добрый вечер. С друзьями можно здороваться как угодно, но «привет» самый правильный вариант;
\item К незнакомым людям обращаться на «вы». На «ты» можно обращаться к другу, родственнику или к человеку, который сам вас об этом попросил. В официальной обстановке на «вы» нужно общаться со всеми;
\item Не называть человека по фамилии. Ровесника по имени, старшего по имени отчеству;
\item При завершении разговора прощаться, используя слова: до свидания, пока, до встречи. Будет уместно сказать, что общение понравилось, что было приятно провести время с человеком;
\item Не перебивать. Если появились вопросы, дослушайте собеседника до конца, быть может, он ответит на вопрос. Если нет, то задавать после паузы. Не перебивайте собеседника, чтобы рассказать похожий случай, приключившейся с вами. Если человек долго говорит, а у вас нет времени дослушать или вы чувствуете, что собеседник может продолжать еще долго, вежливо остановите его сказав, что послушали бы еще, но вам нужно бежать. Извинитесь за то, что перебили. Если собеседник потерял нить разговора, можете сказать, что он уклонился от темы;
\item Если нужно задать вопрос незнакомому человеку, говорите «извините пожалуйста» или «вы не могли бы сказать…». При любом ответе поблагодарите человека;
\item Первым протягивать руку для рукопожатия должен старший или человек с высшей должности.
\end{itemize}

\end{document}