\documentclass[a4paper, 12pt]{article}%тип документа

%%%Библиотеки
	%\usepackage[warn]{mathtext}	
	\usepackage[T2A]{fontenc} % кодировка
	\usepackage[utf8]{inputenc} % кодировка исходного текста
	\usepackage[english,russian]{babel} % локализация и переносы
	\usepackage{caption}
	\usepackage{listings}
	\usepackage{amsmath,amsfonts,amssymb,amsthm,mathtools}
	\usepackage{wasysym}
	\usepackage{graphicx}%Вставка картинок правильная
	\usepackage{float}%"Плавающие" картинки
	\usepackage{wrapfig}%Обтекание фигур (таблиц, картинок и прочего)
	\usepackage{fancyhdr} %загрузим пакет
	\usepackage{lscape}
	\usepackage{xcolor}
	\usepackage[normalem]{ulem}
	\usepackage{hyperref}

%%%Конец библиотек




%%%Настройка ссылок
	\hypersetup
	{
		colorlinks=true,
		linkcolor=blue,
		filecolor=magenta,
		urlcolor=blue
	}
%%%Конец настройки ссылок


%%%Настройка колонтитулы
	\pagestyle{fancy}
	\fancyhead{}
	\fancyhead[L]{Доклад}
	\fancyhead[R]{Талашкевич Даниил, группа Б01-008}
	\fancyfoot[C]{\thepage}
%%%конец настройки колонтитулы



							\begin{document}
						%%%%Начало документа%%%%


%%%Начало титульника
\begin{titlepage}

	\newpage
	\begin{center}
		\normalsize Московский физико-технический институт \\(госудраственный 			университет)
	\end{center}

	\vspace{6em}

	\begin{center}
		\Large Доклад по аналоговой электронике\\
	\end{center}

	\vspace{1em}

	\begin{center}
		\large \textbf{Социально-философские аспекты бизнеса и этика предпринимательства}
	\end{center}

	\vspace{2em}

	\begin{center}
		\large Талашкевич Даниил Александрович\\
		Группа Б01-008
	\end{center}

	\vspace{\fill}

	\begin{center}
	Долгопрудный \\2022
	\end{center}
	
\end{titlepage}
%%%Конец Титульника



%%%Настройка оглавления и нумерации страниц
	\thispagestyle{empty}
	\newpage
	\tableofcontents
	\newpage
	\setcounter{page}{1}
%%%Настройка оглавления и нумерации страниц


					%%%%%%Начало работы с текстом%%%%%%

\section{Вступление, исторические сведения}

Начатая в конце 1980-х гг. реформа в экономике, приватизация 1990-х гг. и нарастание финансовых активов крупных российских предприятий в последние пять-десять лет обусловили появление нового вида мотивации к труду, суть которого состоит в предоставлении экономической самостоятельности субъектам хозяйствования, способным получить максимальную прибыль при развитии своего частного дела - бизнеса посредством пользования имуществом и/или нематериальными активами, продажи товаров, выполнения работ или оказания услуг лицами, зарегистрированными в этом качестве в установленном законом порядке. Предпринимательство и бизнес являются главнейшими атрибутами рыночной экономики, пронизывающими все ее институты.

Важнейшим условием активного роста предпринимательства в России явилась трансформация института собственности, лежащая в основе приватизации, которая сопровождалась невиданным ранее высвобождением энергии, интеллектуального потенциала, менеджерского таланта самой активной части общества, ставшей впоследствии слоем предпринимателей и бизнесменов.

Социально-философский анализ предпринимательства как сопутствующего явления рыночной экономики, ставшего в ХХ веке одним из основных ресурсов наряду с землей, трудом, капиталом, информацией и временем, опирается на исследования австрийской школы экономистов, таких как Л. фон Мизес, Ф.А. фон Хайек и др., выступивших с критикой социалистической формы экономики, находящейся в полной зависимости от государства. Так, главным аргументом Хайека было выявление противоречия социализма, коллективизма, системы плановой экономики принципам правового государства и личного права. Причины варварства и насилия тоталитарных режимов первой половины $\mathrm{XX}$ века в Германии, Италии и Советском Союзе заключались, по его мнению, не в особой агрессивности населения этих стран, а в осуществлении социалистического учения плановой экономики, которая неизбежно вела к угнетению и подавлению, даже если это и не было изначальной целью приверженцев социализма.

В это же время американский экономист Й. Шумпетер предвосхитил актуальное понимание предпринимателя как человека, пытающегося превратить новую идею или изобретение в успешную инновацию. Он также обнаружил ведущую тенденцию современной экономической жизни общества, выразив ее в принципе «креативного разрушения». В результате его идеи легли в основу понимания предпринимательства как особого вида и типа инновационного поведения. Различие между предпринимателями и остальными людьми определяется не столько соответствующей сферой деятельности, сколько «инновационным типом личности».

В российской традиции философии хозяйства феномен предпринимательства анализируется не только в связи с экономическими предпосылками формирования рыночных отношений в постсоветский период, но также определяется в контексте становления особой культуры распоряжения и управления частной собственностью, характеризующей вестернизационный путь развития российского хозяйства. В этом случае роль «капиталиста» занимает «предприниматель», а выгода трактуется как общественная польза.

Социологическое направление в отечественной философии хозяйства подчеркивает связь реформирования общественно-экономической системы с выделением предпринимательства как социального субъекта, обладающего новаторским характером деятельности. Представители этого подхода предлагают расширенную трактовку предпринимательства как «типа общественной деятельности, способствующего рыночной трансформации российской экономики» [1, с. 10], и, следовательно, позволяют считать новаторами всех активных участников становления, функционирования и развития рыночных отношений.
Социально-новаторская деятельность предпринимательства, в рамках
социологического подхода, не сводится лишь к экономической сфере. Она обусловливает преобразование всей системы социальных отношений, формирование новых общественных связей, появление новых стереотипов не только экономического, но и социального, и политического поведения. Для этого подхода характерен серьезный анализ «проблемы применения» классической теории предпринимательства к российскому обществу, но его представители не разработали критериев отличия того, какие типы деятельности и формы поведения соответствуют предпринимательству и что отличает его от других участников рыночной экономики.

\section{Текущая ситуации}

Анализ предпринимательства в современных условиях позволяет выявить родовые, сущностные характеристики этого вида деятельности и определить социальные и индивидуальные черты личности предпринимателя. Не случайно в русском языке слово «предприниматель» не является тождественным его английскому эквиваленту «бизнесмен». Например, весьма распространенный негативный оттенок имеют понятия «грязный бизнес», «теневой бизнес», «наркобизнес», однако подобные выражения с понятием «предпринимательство» не употребляются столь широко. В общем, объединяющим смысловым значением «предпринимательства» и «бизнеса» является целенаправленная деятельность по получению прибыли. Если «бизнесом» именуют любое дело, то «предпринимательством» называют дело, начатое по собственной инициативе, на основе поиска, творческого подхода и риска. Следовательно, предприниматель - не просто «человек дела», но личность, способная находить и принимать нестандартные решения, обеспечивающие освоение новых сфер деятельности, позволяющие в конечном счете расширить человеческие возможности.

В итоге, такие исследователи, как В.В. Радаев, Т.И. Заславская и др., приходят к выводу о том, что предпринимательство образовало новый социальный слой, именуемый бизнесом. При этом бизнес-слой отличается по своим социальным функциям от других социальных слоев точно так же, как частный сектор экономики от государственного. Главная функция этого слоя состоит в управлении социально-экономическими процессами на основе оптимизации используемых ресурсов и получения общественно значимой выгоды. Этот слой в России является главным субъектом перевода народного хозяйства в режим рыночной экономики. Концепция «бизнес-слоя» предполагает анализ составляющих его социальных групп и выделение тех из них, которые подхватывают эстафету лидерства на каждой фазе развития экономики.

В философии хозяйства В. Радаев обосновал эффективность и эмпирических исследований российского предпринимательства, связав его с реализацией такой важной социальной функции, как организационно-хозяйственная инициативность. В этом он был близок к позиции Й. Шумпетера, определяя предпринимателя как экономического субъекта, производящего новые комбинации факторов производства [2, с. 38]. Смысл же предпринимательской деятельности, по его мнению, заключен в создании новых хозяйственных структур, удовлетворяющих потребности общества и одновременно приносящих выгоды их создателю.

Другие исследователи, в частности Э. Фетисов, И. Яковлев, акцентируют внимание на «революционной сущности» социальной роли предпринимателя [3, с. 27-29], которому в силу слабой связи с традицией присуща духовная раскрепощенность, большой запас сил и энергии, способность увлекать людей за собой. Недостаток информации и времени компенсируется особым чутьем или интуицией, позволяющими быстро принимать решения. Эти ученые отождествляют предпринимательство с инновационным типом поведения: внедрением технологических и организационно-хозяйственных новшеств, принятием нестандартных решений, изобретательностью, находчивостью и т.п. При таком подходе различие между предпринимателями и остальными людьми определяется не сферой деятельности или социально-экономическим статусом, a, прежде всего, инновационным типом поведения.

Представители социологического подхода в философии хозяйства при анализе предпринимательства делают акцент скорее на изменениях социальной структуры, чем на социальной субъектности. Так, Т.И. Заславская специально вводит понятие «бизнес-слой» для обозначения нового элемента социальной структуры [4, с. 21]. Ядром этого слоя, по ее мнению, являются предприниматели. Это собственники-владельцы, лично управляющие принадлежащими им предприятиями и не занятые по найму. Но в предпринимательский слой включаются и другие социальные группы, занятые различными видами самодеятельности и существенно различающиеся своим социальным статусом, например, юристы, врачи, журналисты, преподаватели, ученые и т.п. Основанием отнесения этих социальных категорий к предпринимательству является наличие у них уникальных знаний и навыков, которые позволяют им быть относительно самостоятельными и независимыми субъектами на рынке труда, свободно предлагая свои услуги обществу. Причем государство в принципе не в состоянии их проконтролировать. Если государство идет наперекор интересам этих людей, то его законы попросту игнорируются без всяких видимых для данных субъектов последствий.

Для обозначения этих различных социальных групп, «работающих» на развитие рыночных отношений, Т.И. Заславская и вводит специальный термин «бизнес-слой». Это родовое понятие, охватывающее всех субъектов рыночных отношений: от собственников крупных заводов, банков и бирж до тех, кто в свободное время пытается «делать деньги» на свой страх и риск. «Бизнес-слой» представлен совокупностью субъектов производственной, коммерческой и финансовой деятельности, осуществляемой с целью получения прибыли на базе автономно принимаемых решений и под личную ответственность субъектов данной деятельности. Это те субъекты, которые постоянно инициируют новый тип социальности и новый тип субъекта и, таким образом, трансформируют его социальную структуру.

Укрепление социального статуса «бизнес-слоя» и доступность для него основных ресурсов обусловили интенсивность процесса институционализации предпринимательства. В настоящее время предпринимательство и бизнес являются одними из важнейших социально-хозяйственных институтов, изменивших структуру и состав российского общества. В этой связи развитие института предпринимательства как важнейшего субъекта социальных, хозяйственных, политических процессов модернизации экономики приводит к необходимости регулирования совокупности моральных и правовых норм, на основе которых этот институт вступает в социальные отношения с другими.

Процесс институционализации предполагает теоретическое осмысление новой практики деятельности людей и создание таких моральных, правовых и политических норм, которые способствуют оптимизации деятельности учреждений и организаций по удовлетворению сформировавшихся и осознанных людьми новых социальных потребностей и интересов, характерных для всего общества, отдельных социальных групп и индивидов.

Все социальные институты представляют собой совокупность учреждений, соответствующих социальной структуре общества; они основывают свою деятельность на имеющихся и разделяемых большинством населения социальных, в том числе религиозных, моральных, правовых и политических нормах, а также на культурных традициях и стандартах поведения, одобренных этими нормами. Деятельность в рамках социального института предпринимательства предполагает реальное поведение людей в процессе производства и в разных формах сервисного обслуживания в соответствии с конкретными нормами, которые обеспечиваются не только законом, но и общественным мнением, социальным контролем в форме одобрения или неодобрения со стороны членов данного социального института.

В исследовании современного польского философа М. Суханека, посвященном анализу предпринимательской деятельности в контексте философии хозяйства, отмечается взаимосвязь института предпринимательства с появлением в обществе интересов и потребностей людей под влиянием новых целей и интегративных идеалов, которые консолидируют общество и формируют нормы, стандарты поведения и системы ценностей, обеспечивающие нормальное функционирование предпринимательства. Он убежден, что «процесс институционализации предпринимательства - это становление и упрочение конкретных социальных норм, действующих в обществе и обеспечивающих его сохранение, причем радикальные изменения в социальной практике, как это наблюдалось в условиях современной модернизации, ведут к радикальным изменениям и трансформациям внутри традиционно существующих социальных институтов, а также и к созданию новых, которые возникают под влиянием осознания людьми новых потребностей и интересов, на основе чего в постсоциалистических странах возникают новые социальные институты, главным из которых становится институт предпринимательства» $[5$, с. 187]. Социальные институты, являясь формами социальной практики, становятся значимой частью общества, когда они закрепляют свою деятельность в нормах морали и права, а также в конкретных организованных, преимущественно правовых, формах, санкционированных государством.

\section{Вывод}

Таким образом, социальные институты, в частности институт предпринимательства, берут на себя определенную социальную ответственность, что предполагает добровольное выполнение деятельности во благо общества. Такой принцип можно рассматривать как моральную зрелость социального института.

Итак, экономические отношения всегда содержат этическое, нравственное начало, т.к. нравственность имманентно включена в любую сферу деятельности и общения людей. В условиях перехода к рыночным отношениям при создании новых предприятий приоритет должен отдаваться таким, в основе деятельности которых лежат общечеловеческие принципы добра, справедливости, милосердия, гуманизма. Экономика и хозяйство, ориентированные на человека, а не на план, создают новые моральные критерии и оценки, новые этические представления, способствующие нравственному оздоровлению сначала экономики, а потом и всего общества, выражающего готовность к нравственной легитимации предпринимательства и бизнеса.

\section{Литература}

\begin{itemize}

\item Шкаратан М.О. Феномен предпринимателя: интерпретация понятий. М., 2005.

\item Радаев В.В. Новое российское предпринимательство в оценках экспертов // Мир России. 2004. № 1.

\item Фетисов Э.Н., Яковлев И.Г. О социальных аспектах предпринимательства (концептуальное введение в проблему) // СОЦИС. 2003. № 1.

\item Заславская Т.И. Бизнес-слой российского общества: сущность, структура, статус // Общественные науки и современность. 2001. № 1.

\item Суханек М. Предпринимательская деятельность: социально-философский анализ. M., 2003.

\end{itemize}

\end{document}