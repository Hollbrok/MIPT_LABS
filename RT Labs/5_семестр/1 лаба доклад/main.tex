\documentclass[a4paper, 12pt]{article}%тип документа

%%%Библиотеки
	%\usepackage[warn]{mathtext}	
	\usepackage[T2A]{fontenc} % кодировка
	\usepackage[utf8]{inputenc} % кодировка исходного текста
	\usepackage[english,russian]{babel} % локализация и переносы
	\usepackage{caption}
	\usepackage{listings}
	\usepackage{amsmath,amsfonts,amssymb,amsthm,mathtools}
	\usepackage{wasysym}
	\usepackage{graphicx}%Вставка картинок правильная
	\usepackage{float}%"Плавающие" картинки
	\usepackage{wrapfig}%Обтекание фигур (таблиц, картинок и прочего)
	\usepackage{fancyhdr} %загрузим пакет
	\usepackage{lscape}
	\usepackage{xcolor}
	\usepackage[normalem]{ulem}
	\usepackage{hyperref}

%%%Конец библиотек




%%%Настройка ссылок
	\hypersetup
	{
		colorlinks=true,
		linkcolor=blue,
		filecolor=magenta,
		urlcolor=blue
	}
%%%Конец настройки ссылок


%%%Настройка колонтитулы
	\pagestyle{fancy}
	\fancyhead{}
	\fancyhead[L]{Доклад}
	\fancyhead[R]{Талашкевич Даниил, группа Б01-008}
	\fancyfoot[C]{\thepage}
%%%конец настройки колонтитулы



							\begin{document}
						%%%%Начало документа%%%%


%%%Начало титульника
\begin{titlepage}

	\newpage
	\begin{center}
		\normalsize Московский физико-технический институт \\(госудраственный 			университет)
	\end{center}

	\vspace{6em}

	\begin{center}
		\Large Доклад по аналоговой электронике\\
	\end{center}

	\vspace{1em}

	\begin{center}
		\large \textbf{Квантовая запутанность}
	\end{center}

	\vspace{2em}

	\begin{center}
		\large Талашкевич Даниил Александрович\\
		Группа Б01-008
	\end{center}

	\vspace{\fill}

	\begin{center}
	Долгопрудный \\2022
	\end{center}
	
\end{titlepage}
%%%Конец Титульника



%%%Настройка оглавления и нумерации страниц
	\thispagestyle{empty}
	\newpage
	\tableofcontents
	\newpage
	\setcounter{page}{1}
%%%Настройка оглавления и нумерации страниц


					%%%%%%Начало работы с текстом%%%%%%

\section{Вступление}

Нобелевская премия по физике в этом году (4 октября) за «эксперименты с запутанными фотонами, установление нарушения неравенства Белла и новаторскую квантовую информатику» получили Ален Аспект (Франция), Джон Клаузер (США) и Антон Цайлингер (Австрия). То есть они экспериментальным образом доказали существование квантовой запутанности, что дало сильный толчок для развития квантовых технологий: квантовые вычисления и квантовый компьютер, квантовую криптографию, квантовую телепортацию, квантовую метрологию, квантовые сенсоры, квантовые изображения и другие.

В данном докладе освещается суть этого явления и его роль для создания новых технологий.
Помимо этого в конце вы узнаете ответ на важный философский вопрос "предсказуем наш мир или нет, существует ли судьба".

\section{Теоретическое положение}
 
\subsection{История}

Альберт Эйнштейн критиковал эту теорию: ведь способность частиц моментально «угадывать» состояние друг друга означала бы, что они обмениваются информацией быстрее скорости света, что противоречит постулатам теории относительности. По мнению Эйнштейна, должны были существовать некие скрытые параметры, узнав которые, ученые смогли бы вернуть квантовую теорию в русло детерминизма, то есть классической модели. А чтобы найти такие параметры, нужно было бы найти другие составляющие двухчастной системы, которые бы не меняли свои свойства при измерении, в отличие от запутанных частиц.

Джон Стюарт Белл, работавший над этой проблемой, в 1960-х годах века предложил проверить наличие скрытых параметров при помощи неравенства (которое сейчас называется теоремой Белла). По замыслу ученого, если неравенство выполняется, значит, в системе есть скрытые параметры. Доказать это могли бы статистические эксперименты: в случае наличия или отсутствия скрытых параметров вероятность состояний будет отличаться.

Недостаток теории заключался в том, что для ее доказательства необходимо было бы провести тысячи экспериментов, чтобы собрать достаточно статистических данных. Это стало возможно только сильно позже, когда появилось оборудование для фиксации состояния экспериментальных фотонов.

Американский физик Джон Клаузер предложил эксперимент для проверки неравенства Белла, благодаря которому ему в 1972 году удалось доказать, что неравенства не выполняются, а значит, скрытых параметров нет.

Однако работа на этом не завершилась. Клаузер и другие ученые продолжили искать ответы на некоторые спорные моменты.

\subsection{Интерпретация явления}

\textbf{Что такое квантовая запутанность простыми словами}

 Давайте для начала поясним, что такое запутанное состояние (или перепутанное, сцепленное, связанное, переплетенное). Начнем с того, что квантовая запутанность — это всегда история о двухчастичной системе — один атом или фотон не может быть перепутанным. Система может быть и многочастичной, но тогда для определения ее запутанности все равно нужно разделить ее на две подсистемы и рассматривать их корреляции.
 
 Чтобы частицы были перепутанными, они должны были когда-то провзаимодействовать. Если они никогда не взаимодействовали, значит, они не перепутаны. Как пример: две частицы образовались в результате распада одной частицы. Но дальше они физически не взаимодействуют, никаких сил между ними нет. Просто их состояние таково, что они проявляют корреляции в разных измерениях, которые нельзя описать с точки зрения классической физики.

\textbf{Выражение явления математическим уравнением}

Квантовые состояния описываются волновой функцией. Соответственно, если у нас есть две системы, то они описываются совместной волновой функцией $\psi(x_1, x_2)$, которая зависит от параметров первой системы ($x_1$) и второй системы ($x_2$). И если эту совместную волновую функцию нельзя представить в виде произведения волновых функций ее подсистем, то такое состояние называется запутанным. Физически это означает, что параметры этих систем связаны друг с другом. И если я измеряю параметр одной системы, то я сразу получаю информацию о параметре другой системы.

Важно отличать перепутанные состояния от состояний, проявляющих классические корреляции.

\textbf{Пример классической корреляции}: 
У нас есть пьяный стрелок с двуствольным ружьем. Он случайным образом палит во все стороны. И понятно, что каждая пуля — независимо от того, из какого дула она вышла, — может попасть в любую сторону. Но поскольку стрелок одновременно выпускает две пули, то куда пошла одна пуля, туда же примерно пойдет и другая. Эти пули друг с другом более-менее связаны, и если я измерю координаты одной, то примерно смогу понять, куда попала другая.

\textbf{Пример неклассической корреляции}:
Представьте, что мы с сапогами можем проводить некий другой тип измерения, который одинаково — что для правого, что для левого — давал бы с равной вероятностью разные результаты. Например, я могу брать сапог, подбрасывать и смотреть, куда он упадет: направо или налево — так раньше гадали. И вот мои экспериментаторы на Марсе и Венере так же сапоги подбрасывают, и если их сапоги падают в одну и ту же сторону, то получается, что результаты их измерений связаны независимо от того, какой тип измерений они проводят. Вот такие неклассические корреляции и называются запутанностью.

\textbf{Зачем вся эта запутанность вообще была нужна и что она порождает?}

Начнем с фундаментальной истории, которая называется «проверка нарушений неравенства Белла». Что это такое? Существует глубокий философский вопрос о том, предсказуем наш мир или нет, принцип детерминизма. Можно ли определить, как все дальше будет развиваться, или это невозможно и есть принципиальная неопределенность? Долгое время разные ученые и философы считали, что мир предсказуем. Грубо говоря, еще в школе нас учили: если мы кинем шар под углом к горизонту с такой-то скоростью, траекторию можно посчитать. А когда человечество столкнулось с квантовой физикой, выяснилось, что квантовая теория не дает ответа на вопрос, как будет вести себя результат измерения в каждом конкретном эксперименте.

\textbf{Практическое применение квантовой запутанности}

Если говорить про прикладные применения запутанности, то это квантовая метрология, квантовая связь и квантовые вычисления. В разных метрологических приложениях использование перепутанных частиц позволяет точнее измерять время, расстояние, электрические и гравитационные поля и пр.

Перепутанность является ресурсом в квантовых вычислениях. Дело в том, что если у нас есть многочастичная (например, многофотонная) система, то в общем случае состояние такой системы будет перепутано. Оказывается, что для описания такого состояния нужно очень много информации. Если я увеличиваю число квантовых битов, у меня количество коэффициентов будет расти как $2n$. То есть это очень быстрый экспоненциальный рост. (Сейчас в связи с ковидной пандемией мы все усвоили, что такое экспоненциальный рост, когда количество зараженных раз в две-три недели удваивалось.) Таким образом, если у меня будет хотя бы $50-60$ квантовых битов, то мне никакого компьютера не хватит, чтобы записать туда их состояние.

\section{Выводы}

Помимо сильного толчка в развитии квантовых технологий, экспериментальное подтверждение квантовой запутанности ответило на важный философский вопрос "предсказуем наш мир или нет". Нет ! 


\end{document}