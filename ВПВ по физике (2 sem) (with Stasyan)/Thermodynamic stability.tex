\documentclass[a4paper,14pt]{article} % тип документа


%%%Библиотеки
%\usepackage[warn]{mathtext}
%\usepackage[T2A]{fontenc} % кодировка
\usepackage[utf8]{inputenc} % кодировка исходного текста
\usepackage[english,russian]{babel} % локализация и переносы
\usepackage{caption}
\usepackage{listings}
\usepackage{amsmath,amsfonts,amssymb,amsthm,mathtools}
\usepackage{wasysym}
\usepackage{graphicx}%Вставка картинок правильная
\usepackage{float}%"Плавающие" картинки
\usepackage{wrapfig}%Обтекание фигур (таблиц, картинок и прочего)
\usepackage{fancyhdr} %загрузим пакет
\usepackage{lscape}
\usepackage{xcolor}
\usepackage[normalem]{ulem}
\usepackage{hyperref}

%%%Конец библиотек




%%%Настройка ссылок
\hypersetup
{
colorlinks=true,
linkcolor=blue,
filecolor=magenta,
urlcolor=blue
}
%%%Конец настройки ссылок


%%%Настройка колонтитулы
	\pagestyle{fancy}
	\fancyhead{}
	\fancyhead[L]{Вопрос по выбору}
	\fancyhead[R]{Талашкевич Даниил, группа Б01-009}
	\fancyfoot[C]{\thepage}
%%%конец настройки колонтитулы



\begin{document}
%%%%Начало документа%%%%


%%%Начало титульника
\begin{titlepage}

	\newpage
	\begin{center}
		\normalsize Московский физико-технический институт \\(госудраственный 			университет)
	\end{center}

	\vspace{6em}

	\begin{center}
		\Large Устный экзамен по физике\\Вопрос по выбору
	\end{center}

	\vspace{1em}

	\begin{center}
		\large \textbf{Thermodynamic stability}
	\end{center}

	\vspace{2em}

	\begin{center}
		\large X\\
		Группа Б01-X
	\end{center}

	\vspace{\fill}

	\begin{center}
	Долгопрудный \\2021
	\end{center}
	
\end{titlepage}
%%%Конец Титульника



%%%Настройка оглавления и нумерации страниц
\thispagestyle{empty}
\newpage
\tableofcontents
\newpage
\setcounter{page}{1}
%%%Настройка оглавления и нумерации страниц


%%%%%%Начало работы с текстом%%%%%%

\section{Условие а)}

\begin{equation*}
\left(\frac{\partial^{2} U}{\partial S^{2}}\right)_{V}=\left(\frac{\partial T}{\partial S}\right)_{V}=\frac{T}{C_{V}}>0, \quad \text { т. e. } \quad C_{V}>0
\end{equation*}

\section{Условие б)}

\begin{equation}
\begin{aligned}
X=\left(\frac{\partial^{2} U}{\partial S^{2}}\right)_{V}\left(\frac{\partial^{2} U}{\partial V^{2}}\right)_{S} &-\left(\frac{\partial}{\partial V}\left(\frac{\partial U}{\partial S}\right)_{V}\right)\left(\frac{\partial}{\partial S}\left(\frac{\partial U}{\partial V}\right)_{S}\right)=\\
&=-\left(\frac{\partial T}{\partial S}\right)_{V}\left(\frac{\partial P}{\partial V}\right)_{S}+\left(\frac{\partial T}{\partial V}\right)_{S}\left(\frac{\partial P}{\partial S}\right)_{V}>0 .
\end{aligned}
\label{1for}
\end{equation}

Рассматривая давление как функцию объема и температуры $P=P(V, T)$ имеем $d P=(\partial P / \partial V)_{T}dV + \left(\partial P / \partial T)_{V} d T\right.$, откуда

\begin{equation*}
\left(\frac{\partial P}{\partial V}\right)_{S}=\left(\frac{\partial P}{\partial V}\right)_{T}+\left(\frac{\partial P}{\partial T}\right)_{V}\left(\frac{\partial T}{\partial V}\right)_{S}.
\end{equation*}

Подстановка последнего равенства в (\ref{1for}) дает

\begin{equation*}
\begin{aligned}
&X=-\left(\frac{\partial T}{\partial S}\right)_{V}\left[\left(\frac{\partial P}{\partial V}\right)_{T}+\left(\frac{\partial P}{\partial T}\right)_{V}\left(\frac{\partial T}{\partial V}\right)_{S}\right]+ \\
&\ \ \ \ \ \ \ \quad+\left(\frac{\partial T}{\partial V}\right)_{S}\left(\frac{\partial P}{\partial S}\right)_{V}=-\left(\frac{\partial T}{\partial S}\right)_{V}\left(\frac{\partial P}{\partial V}\right)_{T}- \\
&\ \ \ \ \ \ \ \ \ \ \ \ \ -\left(\frac{\partial T}{\partial S}\right)_{V}\left(\frac{\partial P}{\partial T}\right)_{V}\left(\frac{\partial T}{\partial V}\right)_{S}+\left(\frac{\partial T}{\partial V}\right)_{S}\left(\frac{\partial P}{\partial S}\right)_{V} .
\end{aligned}
\end{equation*}

Имея в виду, что
$$
\left(\frac{\partial T}{\partial S}\right)_{V}=\frac{T}{C_{V}}, \quad\left(\frac{\partial T}{\partial S}\right)_{V}\left(\frac{\partial P}{\partial T}\right)_{V}=\left(\frac{\partial P}{\partial S}\right)_{V},
$$
получим
$$
X=-\frac{T}{C_{V}}\left(\frac{\partial P}{\partial V}\right)_{T}>0 .
$$
Вследствие неравенства $C_{V}>0$ получаем, что $(\partial P / \partial V)_{T}<0 .$ Таким образом, независимо от уравнения состояния вещества изотермическая сжимаемость

\begin{equation*}
\beta_{T}=-\frac{1}{V}\left(\frac{\partial V}{\partial P}\right)_{T}>0.
\end{equation*}

Поскольку 

\begin{equation*}
C_{P}-C_{V}=-T \frac{(\partial V / \partial T)_{P}^{2}}{(\partial V / \partial P)_{T}},
\end{equation*}

то из полученного неравенства следует, что всегда $C_{P}>C_{V} .$ Имея в виду также, что $C_{V}>0$, заключаем, что показатель адиабаты $\gamma=C_{P} / C_{V}>1 .$ Для положительной определенности квадратичной формы в (1.5.1\textbf{(СДЕЛАТЬ ССЫЛКУ НА ФОРМУЛУ СТАСИКА)}) можно было бы условие \textbf{а)} заменить условием $\left(\partial^{2} U / \partial V^{2}\right)_{S}>0$ или $\left(\partial^{2} U / \partial V^{2}\right)_{S}=-(\partial P / \partial V)_{S}>0 .$ Последнее означает, что адиабатическая сжимаемость также положительна:

\begin{equation*}
\beta_{T}=-\frac{1}{V}\left(\frac{\partial V}{\partial P}\right)_{T}>0.
\end{equation*}

Условия термодинамической устойчивости $C_V > 0$ и $\beta_{T} > 0$ называют 
термодинамическими неравенствами.

\section{Смысл условий устойчивости}

\hskip 5mm Предположим, что подсистема находится в тепловом и механическом равновесии с внешней средой, т. е. $T=T_{0},\ P=P_{0} .$ Покажем, что при нарушении найденных условий состояние равновесия не может быть устойчивым.

1) Допустим, что $C_{V}<0 .$ Пусть температура подсистемы случайно уменышилась, $T<T_{0} .$ Тогда в соответствии со вторым началом термодинамики в эту подсистему потечет тепловой поток из внешней среды. Поскольку $\delta Q=C_{V} d T>0$, то в результате температура $T$ еще более уменьшится. Аналогично, случайное увеличение температуры подсистемы приведет к ее дальнейшему увеличению. Следовательно, тепловое равновесие неустойчиво.

2) Допустим, что $(\partial \boldsymbol{P} / \partial \boldsymbol{V})_{\boldsymbol{T}}>\mathbf{0 .}$ Пусть объем подсистемы случайно уменьшился. Тогда давление в ней также уменьшилось, $P<P_{0} .$ В результате внешнее давление $P_{0}$ оказывается больше, чем внутреннее. Поэтому объем подсистемы будет и дальше уменьшаться. Аналогично, при случайном увеличении объема подсистемы ее объем будет продолжать увеличиваться. Следовательно, механическое равновесие оказывается неустойчивым.

\end{document}